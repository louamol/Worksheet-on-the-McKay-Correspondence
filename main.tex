\documentclass{worksheetclass}

\usepackage{import}
\import{}{custom_macros.tex}

\title{The McKay Correspondence}

% DOCUMENT -----------------------------

\begin{document}

\maketitle

\tableofcontents

\section{McKay correspondence}\label{app:McKay}

    \subsection{Classical correspondence}

        \begin{table}[H]
            \centering
            \begin{tabular}{|c|l|l|l|}
                \hline
                $\Gamma\subset\SU(2)$ & Platonic solids & McKay graph & Algebraic variety \\ \hline
                $\Z_n$ &  & \begin{tikzpicture}[baseline={($ (current bounding box.center) - (0,3pt) $)},scale=0.5]
                    \draw (0,0) edge (2*1.25,0);
                    \draw (2*1.25,0) edge[dashed] (3*1.25,0);
                    \draw (3*1.25,0) edge (4*1.25,0);
                    \draw (0,0) edge (2*1.25,-1);
                    \draw (4*1.25,0) edge (2*1.25,-1);
                    \foreach \x in {0,1,2,3,4} {
                        \draw[fill=black] (1.25*\x,0) circle[radius=0.15];
                        \draw (1.25*\x,0) node[above]{$1$};
                    }
                    \draw[fill=black] (1.25*2,-1) circle[radius=0.15];
                    \draw (1.25*2,-1) node[above]{$1$};
                \end{tikzpicture}\quad($n$ nodes) & $z^{n}+xy=0$ \\ \hline
                $2\mathcal{D}_n$ & $n$-polygon & \begin{tikzpicture}[baseline={($ (current bounding box.center) - (0,3pt) $)},scale=0.5]
                    \draw (0,0) edge (2*1.25,0);
                    \draw (2*1.25,0) edge[dashed] (3*1.25,0);
                    \draw (3*1.25,0) edge (4*1.25,0);
                    \draw (4*1.25,0) edge (5*1.25,0);
                    \draw (1.25,0) edge (1.25,-1.25);
                    \draw (4*1.25,0) edge (4*1.25,-1.25);
                    \foreach \x in {0,1,2,3,4,5} {
                        \draw[fill=black] (1.25*\x,0) circle[radius=0.15];
                    }
                    
                    \draw[fill=black] (1.25,-1.25) circle[radius=0.15];
                    \draw (1.25,-1.25) node[right]{$1$};
                    \draw[fill=black] (4*1.25,-1.25) circle[radius=0.15];
                    \draw (4*1.25,-1.25) node[right]{$1$};
                    \draw (0,0) node[above]{$1$};
                    \draw (1.25*5,0) node[above]{$1$};
                    \draw (1.25*1,0) node[above]{$2$};
                    \draw (1.25*2,0) node[above]{$2$};
                    \draw (1.25*3,0) node[above]{$2$};
                    \draw (1.25*4,0) node[above]{$2$};
                \end{tikzpicture}\quad($n+3$ nodes) & $x^2+y^2z+z^{n-1}=0$ \\ \hline
                $2\mathcal{T}$ & tetrahedron & \begin{tikzpicture}[baseline={($ (current bounding box.center) - (0,3pt) $)},scale=0.5]
                    \draw (0,0) edge (4*1.25,0);
                    \draw (2*1.25,0) edge (2*1.25,-2*1.25);
                    \foreach \x in {0,1,2,3,4} {
                        \draw[fill=black] (1.25*\x,0) circle[radius=0.15];
                    }
                    \draw[fill=black] (2*1.25,-1.25) circle[radius=0.15];
                    \draw (2*1.25,-1.25) node[right]{$2$};
                    \draw[fill=black] (2*1.25,-2*1.25) circle[radius=0.15];
                    \draw (2*1.25,-2*1.25) node[right]{$1$};
                    \draw (0,0) node[above]{$1$};
                    \draw (1.25*4,0) node[above]{$1$};
                    \draw (1.25*1,0) node[above]{$2$};
                    \draw (1.25*2,0) node[above]{$3$};
                    \draw (1.25*3,0) node[above]{$2$};
                \end{tikzpicture}\quad($7$ nodes) & $x^2+y^3+z^4=0$ \\ \hline
                $2\mathcal{O}$  & \begin{tabular}{@{}l@{}}cube \\ octahedron\end{tabular} & \begin{tikzpicture}[baseline={($ (current bounding box.center) - (0,3pt) $)},scale=0.5]
                    \draw (0,0) edge (6*1.25,0);
                    \draw (3*1.25,0) edge (3*1.25,-1.25);
                    \foreach \x in {0,1,2,3,4,5,6} {
                        \draw[fill=black] (1.25*\x,0) circle[radius=0.15];
                    }
                    \draw[fill=black] (3*1.25,-1.25) circle[radius=0.15];
                    \draw (3*1.25,-1.25) node[right]{$2$};
                    \draw (0,0) node[above]{$1$};
                    \draw (1.25*1,0) node[above]{$2$};
                    \draw (1.25*2,0) node[above]{$3$};
                    \draw (1.25*3,0) node[above]{$4$};
                    \draw (1.25*4,0) node[above]{$3$};
                    \draw (1.25*5,0) node[above]{$2$};
                    \draw (1.25*6,0) node[above]{$1$};
                \end{tikzpicture}\quad($8$ nodes) & $x^2+y^3+yz^3=0$ \\ \hline
                $2\mathcal{I}$  & \begin{tabular}{@{}l@{}}icosahedron \\ dodecahedron\end{tabular} & \begin{tikzpicture}[baseline={($ (current bounding box.center) - (0,3pt) $)},scale=0.5]
                    \draw (0,0) edge (7*1.25,0);
                    \draw (2*1.25,0) edge (2*1.25,-1.25);
                    \foreach \x in {0,1,2,3,4,5,6,7} {
                        \draw[fill=black] (1.25*\x,0) circle[radius=0.15];
                    }
                    \draw[fill=black] (2*1.25,-1.25) circle[radius=0.15];
                    \draw (2*1.25,-1.25) node[right]{$3$};
                    \draw (0,0) node[above]{$2$};
                    \draw (1.25*1,0) node[above]{$4$};
                    \draw (1.25*2,0) node[above]{$6$};
                    \draw (1.25*3,0) node[above]{$5$};
                    \draw (1.25*4,0) node[above]{$4$};
                    \draw (1.25*5,0) node[above]{$3$};
                    \draw (1.25*6,0) node[above]{$2$};
                    \draw (1.25*7,0) node[above]{$1$};
                \end{tikzpicture}\quad($9$ nodes) & $x^2+y^3+z^5=0$ \\ \hline
            \end{tabular}
            \caption{Binary polyhedral groups and their McKay graphs.Labels over the vertices are the dimension of the representation. We erase the arrow ends if they go in both directions and erase the label if it is
            equal to $1$.}
        \end{table}

        \begin{figure}[H]
            \centering
            \begin{tabular}{|c|c|l|}
                \hline
                \begin{tabular}{@{}c@{}}Simple \\ Lie algebra\end{tabular} & Simply laced & \begin{tabular}{@{}l@{}}Dynkin diagram \\ Extended Dybkin diagram\end{tabular} \\ \hline
                $\mathfrak{sl}(n+1,\C),n\geq1$ & yes & 
                \begin{tabular}{@{}l@{}} $A_n:\quad$ \begin{tikzpicture}[baseline={($ (current bounding box.center) - (0,3pt) $)},scale=0.5]
                    \draw (0,0) edge (2*1.25,0);
                    \draw (2*1.25,0) edge[dashed] (3*1.25,0);
                    \draw (3*1.25,0) edge (4*1.25,0);
                    \foreach \x in {0,1,2,3,4} {
                    \draw[fill=white] (1.25*\x,0) circle[radius=0.15];
                    }
                    \end{tikzpicture}\quad($n$ nodes) \\[0.4cm] $\tilde{A}_n:\quad$ \begin{tikzpicture}[baseline={($ (current bounding box.center) - (0,3pt) $)},scale=0.5]
                        \draw (0,0) edge (2*1.25,0);
                        \draw (2*1.25,0) edge[dashed] (3*1.25,0);
                        \draw (3*1.25,0) edge (4*1.25,0);
                        \draw (0,0) edge (2*1.25,1);
                        \draw (4*1.25,0) edge (2*1.25,1);
                        \foreach \x in {0,1,2,3,4} {
                            \draw[fill=white] (1.25*\x,0) circle[radius=0.15];
                        }
                        \draw[fill=black] (1.25*2,1) circle[radius=0.15];
                    \end{tikzpicture}\quad($n+1$ nodes)\end{tabular} \\ \hline
                $\mathfrak{so}(2n+1,\R),n\geq2$ & no & 
                \begin{tabular}{@{}l@{}}$B_n:\quad$ \begin{tikzpicture}[baseline={($ (current bounding box.center) - (0,3pt) $)},scale=0.5]
                    \draw (0,0) edge (2*1.25,0);
                    \draw (2*1.25,0) edge[dashed] (3*1.25,0);
                    \draw (3*1.25+0.65-0.15,0.21) -- (3*1.25+0.65+0.15,0) -- (3*1.25+0.65-0.15,-0.21);
                    \draw (3*1.25,0.07) -- (4*1.25,0.07);
                    \draw (3*1.25,-0.07) -- (4*1.25,-0.07); 
                    \foreach \x in {0,1,2,3,4} {
                    \draw[fill=white] (1.25*\x,0) circle[radius=0.15];
                    }
                    \end{tikzpicture}\quad ($n$ nodes) \\[0.4cm] $\tilde{B}_n:\quad$ \begin{tikzpicture}[baseline={($ (current bounding box.center) - (0,3pt) $)},scale=0.5]
                        \draw (1.25,0) edge (2*1.25,0);
                        \draw (0,0.7) edge (1.25,0);
                        \draw (0,-0.7) edge (1.25,0);
                        \draw (2*1.25,0) edge[dashed] (3*1.25,0);
                        \draw (3*1.25+0.65-0.15,0.21) -- (3*1.25+0.65+0.15,0) -- (3*1.25+0.65-0.15,-0.21);
                        \draw (3*1.25,0.07) -- (4*1.25,0.07);
                        \draw (3*1.25,-0.07) -- (4*1.25,-0.07); 
                        \foreach \x in {1,2,3,4} {
                            \draw[fill=white] (1.25*\x,0) circle[radius=0.15];
                        }
                        \draw[fill=white] (0,0.7) circle[radius=0.15];
                        \draw[fill=black] (0,-0.7) circle[radius=0.15];
                    \end{tikzpicture}\quad($n+1$ nodes)\end{tabular} \\ \hline
                $\mathfrak{sp}(2n,\C),n\geq3$ & no & 
                \begin{tabular}{@{}l@{}}$C_n:\quad$ \begin{tikzpicture}[baseline={($ (current bounding box.center) - (0,4pt) $)},scale=0.5]
                    \draw (0,0) edge (2*1.25,0);
                    \draw (2*1.25,0) edge[dashed] (3*1.25,0);
                    \draw (3*1.25+0.65+0.15,0.21) -- (3*1.25+0.65-0.15,0) -- (3*1.25+0.65+0.15,-0.21);
                    \draw (3*1.25,0.07) -- (4*1.25,0.07);
                    \draw (3*1.25,-0.07) -- (4*1.25,-0.07); 
                    \foreach \x in {0,1,2,3,4} {
                    \draw[fill=white] (1.25*\x,0) circle[radius=0.15];
                    }
                    \end{tikzpicture}\quad ($n$ nodes) \\[0.4cm] $\tilde{C}_n:\quad$ \begin{tikzpicture}[baseline={($ (current bounding box.center) - (0,4pt) $)},scale=0.5]
                        \draw (0.65-0.15,0.21) -- (0.65+0.15,0) -- (0.65-0.15,-0.21);
                        \draw (0,0.07) -- (1.25,0.07);
                        \draw (0,-0.07) -- (1.25,-0.07);
                        \draw (1.25,0) edge (2*1.25,0);
                        \draw (2*1.25,0) edge[dashed] (3*1.25,0);
                        \draw (3*1.25+0.65+0.15,0.21) -- (3*1.25+0.65-0.15,0) -- (3*1.25+0.65+0.15,-0.21);
                        \draw (3*1.25,0.07) -- (4*1.25,0.07);
                        \draw (3*1.25,-0.07) -- (4*1.25,-0.07); 
                        \foreach \x in {1,2,3,4} {
                            \draw[fill=white] (1.25*\x,0) circle[radius=0.15];
                        }
                        \draw[fill=black] (0,0) circle[radius=0.15];
                    \end{tikzpicture}\quad($n+1$ nodes)\end{tabular} \\ \hline
                $\mathfrak{so}(2n,\R),n\geq4$ & yes &
                \begin{tabular}{@{}l@{}}$D_n:\quad$ \begin{tikzpicture}[baseline={($ (current bounding box.center) - (0,3pt) $)},scale=0.5]
                    \draw (0,0) edge (2*1.25,0);
                    \draw (2*1.25,0) edge[dashed] (3*1.25,0);
                    \draw (3*1.25,0) -- (4*1.25,0.7);
                    \draw (3*1.25,0) -- (4*1.25,-0.7); 
                    \foreach \x in {0,1,2,3} {
                    \draw[fill=white] (1.25*\x,0) circle[radius=0.15];
                    }
                    \draw[fill=white] (1.25*4,0.7) circle[radius=0.15];
                    \draw[fill=white] (1.25*4,-0.7) circle[radius=0.15];
                    \end{tikzpicture}\quad($n$ nodes) \\[0.4cm] $\tilde{D}_n:\quad$  \begin{tikzpicture}[baseline={($ (current bounding box.center) - (0,3pt) $)},scale=0.5]
                        \draw (0,0.7) edge (1.25,0);
                        \draw (0,-0.7) edge (1.25,0);
                        \draw (1.25,0) edge (2*1.25,0);
                        \draw (2*1.25,0) edge[dashed] (3*1.25,0);
                        \draw (3*1.25,0) -- (4*1.25,0.7);
                        \draw (3*1.25,0) -- (4*1.25,-0.7); 
                        \foreach \x in {1,2,3} {
                            \draw[fill=white] (1.25*\x,0) circle[radius=0.15];
                        }
                        \draw[fill=white] (0,0.7) circle[radius=0.15];
                        \draw[fill=black] (0,-0.7) circle[radius=0.15];
                        \draw[fill=white] (1.25*4,0.7) circle[radius=0.15];
                        \draw[fill=white] (1.25*4,-0.7) circle[radius=0.15];
                    \end{tikzpicture}\quad($n+1$ nodes)\end{tabular} \\ \hline
                $\mathfrak{e}_6$ & yes &
                \begin{tabular}{@{}l@{}}$E_6:\quad$ \begin{tikzpicture}[baseline={($ (current bounding box.south) + (0,1pt) $)},scale=0.5]
                    \draw (0,0) edge (4*1.25,0);
                    \draw (2*1.25,0) edge (2*1.25,1.25);
                    \foreach \x in {0,1,2,3,4} {
                    \draw[fill=white] (1.25*\x,0) circle[radius=0.15];
                    }
                    \draw[fill=white] (2*1.25,1.25) circle[radius=0.15];
                    \end{tikzpicture} \quad($6$ nodes) \\[0.4cm]  $\tilde{E}_6:\quad$ \begin{tikzpicture}[baseline={($ (current bounding box.south) + (0,1pt) $)},scale=0.5]
                        \draw (0,0) edge (4*1.25,0);
                        \draw (2*1.25,0) edge (2*1.25,2*1.25);
                        \foreach \x in {0,1,2,3,4} {
                            \draw[fill=white] (1.25*\x,0) circle[radius=0.15];
                        }
                        \draw[fill=white] (2*1.25,1.25) circle[radius=0.15];
                        \draw[fill=black] (2*1.25,2*1.25) circle[radius=0.15];
                    \end{tikzpicture}\quad($7$ nodes)\end{tabular} \\ \hline
                $\mathfrak{e}_7$ & yes & 
                \begin{tabular}{@{}l@{}} $E_7:\quad$ \begin{tikzpicture}[baseline={($ (current bounding box.south) + (0,1pt) $)},scale=0.5]
                    \draw (0,0) edge (5*1.25,0);
                    \draw (2*1.25,0) edge (2*1.25,1.25);
                    \foreach \x in {0,1,2,3,4,5} {
                    \draw[fill=white] (1.25*\x,0) circle[radius=0.15];
                    }
                    \draw[fill=white] (2*1.25,1.25) circle[radius=0.15];
                    \end{tikzpicture} \quad($7$ nodes) \\[0.4cm] $\tilde{E}_7:\quad$ \begin{tikzpicture}[baseline={($ (current bounding box.south) + (0,1pt) $)},scale=0.5]
                        \draw (-1.25,0) edge (5*1.25,0);
                        \draw (2*1.25,0) edge (2*1.25,1.25);
                        \foreach \x in {0,1,2,3,4,5} {
                            \draw[fill=white] (1.25*\x,0) circle[radius=0.15];
                        }
                        \draw[fill=black] (-1.25,0) circle[radius=0.15];
                        \draw[fill=white] (2*1.25,1.25) circle[radius=0.15];
                    \end{tikzpicture}\quad($8$ nodes)\end{tabular} \\ \hline
                $\mathfrak{e}_8$ & yes & 
                \begin{tabular}{@{}l@{}}$E_8:\quad$ \begin{tikzpicture}[baseline={($ (current bounding box.south) + (0,1pt) $)},scale=0.5]
                    \draw (0,0) edge (6*1.25,0);
                    \draw (2*1.25,0) edge (2*1.25,1.25);
                    \foreach \x in {0,1,2,3,4,5,6} {
                    \draw[fill=white] (1.25*\x,0) circle[radius=0.15];
                    }
                    \draw[fill=white] (2*1.25,1.25) circle[radius=0.15];
                    \end{tikzpicture} \quad($8$ nodes) \\[0.4cm] $\tilde{E}_8:\quad$ \begin{tikzpicture}[baseline={($ (current bounding box.south) + (0,1pt) $)},scale=0.5]
                        \draw (0,0) edge (7*1.25,0);
                        \draw (2*1.25,0) edge (2*1.25,1.25);
                        \foreach \x in {0,1,2,3,4,5,6} {
                            \draw[fill=white] (1.25*\x,0) circle[radius=0.15];
                        }
                        \draw[fill=black] (7*1.25,0) circle[radius=0.15];
                        \draw[fill=white] (2*1.25,1.25) circle[radius=0.15];
                    \end{tikzpicture}\quad($9$ nodes)\end{tabular} \\ \hline
                $\mathfrak{f}_4$ & no & 
                \begin{tabular}{@{}l@{}}$F_4:\quad$ \begin{tikzpicture}[baseline={($ (current bounding box.center) - (0,3pt) $)},scale=0.5]
                    \draw (0,0) edge (1.25,0);
                    \draw (2*1.25,0) edge (3*1.25,0);
                    \draw (1*1.25+0.65-0.15,0.21) -- (1*1.25+0.65+0.15,0) -- (1*1.25+0.65-0.15,-0.21);
                    \draw (1*1.25,0.07) -- (2*1.25,0.07);
                    \draw (1*1.25,-0.07) -- (2*1.25,-0.07); 
                    \foreach \x in {0,1,2,3} {
                    \draw[fill=white] (1.25*\x,0) circle[radius=0.15];
                    }
                    \end{tikzpicture} \quad($4$ nodes) \\[0.4cm] $\tilde{F}_4:\quad$  \begin{tikzpicture}[baseline={($ (current bounding box.center) - (0,3pt) $)},scale=0.5]
                        \draw (-1.25,0) edge (1.25,0);
                        \draw (2*1.25,0) edge (3*1.25,0);
                        \draw (1*1.25+0.65-0.15,0.21) -- (1*1.25+0.65+0.15,0) -- (1*1.25+0.65-0.15,-0.21);
                        \draw (1*1.25,0.07) -- (2*1.25,0.07);
                        \draw (1*1.25,-0.07) -- (2*1.25,-0.07); 
                        \foreach \x in {0,1,2,3} {
                            \draw[fill=white] (1.25*\x,0) circle[radius=0.15];
                        }
                        \draw[fill=black] (-1.25,0) circle[radius=0.15];
                    \end{tikzpicture}\quad($5$ nodes)\end{tabular} \\ \hline
                $\mathfrak{g}_2$ & no & 
                \begin{tabular}{@{}l@{}} $G_2:\quad$  \begin{tikzpicture}[baseline={($ (current bounding box.center) - (0,3pt) $)},scale=0.5]
                    \draw (0,0) edge (1.25,0);
                    \draw (0.65-0.15,0.21) -- (0.65+0.15,0) -- (0.65-0.15,-0.21);
                    \draw (0,0.11) -- (1.25,0.11);
                    \draw (0,-0.11) -- (1.25,-0.11); 
                    \foreach \x in {0,1} {
                    \draw[fill=white] (1.25*\x,0) circle[radius=0.15];
                    }
                    \end{tikzpicture} \quad($2$ nodes) \\[0.4cm] $\tilde{G}_2:\quad$ \begin{tikzpicture}[baseline={($ (current bounding box.center) - (0,3pt) $)},scale=0.5]
                        \draw (-1.25,0) edge (1.25,0);
                        \draw (0.65-0.15,0.21) -- (0.65+0.15,0) -- (0.65-0.15,-0.21);
                        \draw (0,0.11) -- (1.25,0.11);
                        \draw (0,-0.11) -- (1.25,-0.11); 
                        \foreach \x in {0,1} {
                            \draw[fill=white] (1.25*\x,0) circle[radius=0.15];
                        }
                        \draw[fill=black] (-1.25,0) circle[radius=0.15];
                    \end{tikzpicture}\quad($3$ nodes)\end{tabular} \\ \hline
            \end{tabular}
            \caption{Simple Lie algebras and their (extended) Dynkin diagrams. The first four algebras are the classical simple Lie algebras and the last five are the excpetional simple Lie algebras.}
        \end{figure}

        Finally, we can see the following correspondence between the extended Dynkin diagrams and the McKay graphs.

        \begin{figure}[H]
            \centering
            \begin{tabular}{|c|c|c|c|c|}
                \hline
                \begin{tabular}{@{}c@{}} Simply Lie \\ group \end{tabular} & \begin{tabular}{@{}c@{}} Simply laced \\ Lie algebra \end{tabular} & \begin{tabular}{@{}c@{}} Extended \\ Dybkin diagram \end{tabular} & \begin{tabular}{@{}c@{}} Finite subgroup \\ of $\SO(3)$ \end{tabular} & \begin{tabular}{@{}c@{}} Finite subgroup \\ of $\SU(2)$ \end{tabular} \\ \hline
                $\SU(n+1)$ & $\mathfrak{sl}(n+1,\C)$ & $\tilde{A}_n$ & $\Z_{n+1}$ & $\Z_{n+1}$ \\ \hline
                $\SO(2n),\Spin(2n)$ & $\mathfrak{so}(2,\R)$ & $\tilde{D}_n$ & $\D_{2(n-2)}$ & $2\D_{2(n-2)}$ \\ \hline
                $E6$ & $\mathfrak{e}_6$ & $\tilde{E_6}$ &  $\T$ & $2\T$ \\ \hline
                $E7$ & $\mathfrak{e}_7$ & $\tilde{E_7}$ & $\O$ & $2\O$ \\ \hline
                $E8$ & $\mathfrak{e}_8$ & $\tilde{E_8}$ & $\I$ & $2\I$ \\ \hline
            \end{tabular}
            \caption{Classical McKay correspondence.}
        \end{figure}

    \subsection{Geometrical McKay correspondence}

        The geometrical McKay correspondence is the bijection between the exeptional graph of the blow up of orbifolds $\C^2/\Gamma$ ($\Gamma\subset\SU(2)$) and the McKay graphs of $\Gamma$.

    %\subsection{Integral transform interpretation}

        %An \emph{integral transform} on functions is a linear map between the space of fucntions on a space $X$ to the space of functions on some other space $Y$, encoded by some function on the product space $X\times Y$ and that has the form
        %\begin{equation}
        %    \left(
        %        \begin{array}{ccc}
        %            \{f:A\to X\} & \longrightarrow & \{f:A\to Y\} \\
        %            f & \longmapsto & g_f
        %        \end{array}
        %        \right),\qquad g_f(y)=\int_X~K(x,y)f(x).
        %\end{equation}
        %The McKay correspondence can be interpreted in terms of the Fourier-Mukai transform (categorified integral transform).

\printbibliography

\end{document}